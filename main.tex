\documentclass[
paper = a4,
fontsize = 12pt,
numbers=noenddot,
headsepline = true,
footsepline = true,
plainfootsepline = true,
parskip,								        
listof = nottotoc,
bibliography = totoc,
index = totoc,
twoside = false
]{scrartcl}

% -------------------------------------------------
% Pakete: Sprache, Schrift, Layout
% -------------------------------------------------
\usepackage[ngerman]{babel}
\usepackage[T1]{fontenc}
\usepackage[utf8]{inputenc} % bei pdflatex
\usepackage{lmodern}

\usepackage{color} 								% Schrift färben
\usepackage{tikz}
\usetikzlibrary{arrows.meta, calc, positioning, shapes.symbols, shapes.misc}

\usepackage{geometry}
\geometry{
    left=3cm,
    right=2.5cm,
    top=2.5cm,
    bottom=2.5cm
}

\usepackage{setspace}
\onehalfspacing

% -------------------------------------------------
% Mathe, Grafiken, Tabellen
% -------------------------------------------------
\usepackage{amsmath, amssymb}
\usepackage{graphicx}
\usepackage{booktabs}
\usepackage{caption}
\usepackage{subcaption}

% -------------------------------------------------
% Kopf- und Fußzeilen (KOMA-konform, einheitlich)
% -------------------------------------------------
\usepackage{scrlayer-scrpage}
\clearpairofpagestyles

\usepackage{xcolor}
\definecolor{mygray}{rgb}{0.5,0.5,0.5}

% Automatische Marken für scrartcl: section
\automark{section}

% Linienfarbe
\addtokomafont{headsepline}{\color{mygray}}
\addtokomafont{footsepline}{\color{mygray}}

% Kopfzeile: rechts Abschnittsname
\ohead{\color{mygray}\leftmark}

% Fußzeile: Seitenzahl mittig
\cfoot{\color{mygray}\pagemark}

% Auch für plain-Seiten (TOC etc.) erzwingen
\pagestyle{scrheadings}

\setlength{\footskip}{1.2cm}


% -------------------------------------------------
% Titelinformationen
% -------------------------------------------------
\title{
    \textbf{\\[2cm] Dokumentation zum XML-Projekt im Master-Modul Internettechnologien}     \vspace{1cm}
}

\author{
    \Large{\textbf{Lennart Mende}} \\
    \Large{\textbf{Richard Mende}} \\[2cm]
    \large{HTWK Leipzig} \\
    \large{Wintersemester 2025/26} \\
    \large{Prof. Dr.-Ing. Andreas Pretschner}
}

\date{\vspace{3cm}\large{\today}}

% =================================================
\begin{document}
% =================================================

% -----------------------------
% Titelseite
% -----------------------------
\maketitle

\thispagestyle{empty}

\newpage

% -----------------------------
% Inhaltsverzeichnis
% -----------------------------
\tableofcontents

\thispagestyle{empty}
% \vspace{2cm}

% -----------------------------
% Abbildungs- und Tabellenverzeichnis (optional)
% -----------------------------
% \listoffigures

% \vspace{2cm}

% \listoftables
\newpage
\setcounter{page}{1}

% =================================================
% Hauptteil
% =================================================

\section{Zielstellung}

Als Grundlage dieses Projekts dient das Anlegen eines XML-Datensatzes. Dieser sollte aus mehreren Objekten mit jeweils mindestens einem Attribut und einer Datensequenz bestehen. Die daran anschließende Aufgabe besteht im Erstellen einer XSD-Schemadatei. Damit kann validiert werden, ob die gegebene XML-Datei die gewünschte Struktur erfüllt und alle benötigten Angaben gemacht wurden. Darüber hinaus soll eine in \textit{editiX} automatisch generierte XSD-Schemadatei erstellt werden.

Der zweite Teil des Projekts befasst sich mit der Transformation der XML-Datei. Als Erstes sollen die Daten auf einer HTML-Page dargestellt werden. Dazu muss zunächst die XSLT-Transformationsdatei erzeugt werden. Darüber hinaus soll die Transformation in eine PDF erfolgen. Dazu gibt es die Möglichkeit der direkten FOP-Tranformation oder die Docbook-Transformation. Eine übersichtliche Darstellung dieser Transformationen zeigt Abbildung \ref{fig:files}.

Der dritte Aufgabenteil setzt sich mit der XML-Programmierung auseinander. Dabei soll eine Apllikation erstellt werden, welche die Validierung, Abfrage und Transformation der XML-Datei ermöglichen soll.

\enlargethispage{1\baselineskip}

\begin{figure}[htbp]
\centering
\begin{tikzpicture}[>=Stealth, thick, node distance=2.5cm and 3cm]

% ---------------- Parameter ----------------
\def\filewidth{1.8}
\def\fileheight{2.7}
\def\fold{0.5}
\def\scale{1}

% ---------------- XML ----------------
\node (xml) at (0,0) {
\begin{tikzpicture}[x=\scale cm,y=\scale cm]
\draw[fill=white]
(0,0) -- (0,\fileheight) --
(\filewidth-\fold,\fileheight) --
(\filewidth,\fileheight-\fold) --
(\filewidth,0) -- cycle;
\draw[fill=white]
(\filewidth-\fold,\fileheight) --
(\filewidth-\fold,\fileheight-\fold) --
(\filewidth,\fileheight-\fold) -- cycle;
\draw (\filewidth-\fold,\fileheight) -- (\filewidth,\fileheight-\fold);
\node[anchor=center, text width=\filewidth cm, align=center]
at (\filewidth/2,\fileheight/2) {XML};
\end{tikzpicture}
};

% ---------------- XSD ----------------
\node (xsd) [below=5cm of xml] {
\begin{tikzpicture}[x=\scale cm,y=\scale cm]
\draw[fill=white]
(0,0) -- (0,\fileheight) --
(\filewidth-\fold,\fileheight) --
(\filewidth,\fileheight-\fold) --
(\filewidth,0) -- cycle;
\draw[fill=white]
(\filewidth-\fold,\fileheight) --
(\filewidth-\fold,\fileheight-\fold) --
(\filewidth,\fileheight-\fold) -- cycle;
\draw (\filewidth-\fold,\fileheight) -- (\filewidth,\fileheight-\fold);
\node[anchor=center, text width=\filewidth cm, align=center]
at (\filewidth/2,\fileheight/2) {XSD};
\end{tikzpicture}
};

\draw[<->] (xml) -- (xsd);

% ---------------- XSLT ----------------
\node (xslt) [right=of xml] {
\begin{tikzpicture}[x=\scale cm,y=\scale cm]
\draw[fill=white]
(0,0) -- (0,\fileheight) --
(\filewidth-\fold,\fileheight) --
(\filewidth,\fileheight-\fold) --
(\filewidth,0) -- cycle;
\draw[fill=white]
(\filewidth-\fold,\fileheight) --
(\filewidth-\fold,\fileheight-\fold) --
(\filewidth,\fileheight-\fold) -- cycle;
\draw (\filewidth-\fold,\fileheight) -- (\filewidth,\fileheight-\fold);
\node[anchor=center, text width=\filewidth cm, align=center]
at (\filewidth/2,\fileheight/2) {XSLT};
\end{tikzpicture}
};

% ---------------- DocBook ----------------
\node (docbook) [below=5cm of xslt] {
\begin{tikzpicture}[x=\scale cm,y=\scale cm]
\draw[fill=white]
(0,0) -- (0,\fileheight) --
(\filewidth-\fold,\fileheight) --
(\filewidth,\fileheight-\fold) --
(\filewidth,0) -- cycle;
\draw[fill=white]
(\filewidth-\fold,\fileheight) --
(\filewidth-\fold,\fileheight-\fold) --
(\filewidth,\fileheight-\fold) -- cycle;
\draw (\filewidth-\fold,\fileheight) -- (\filewidth,\fileheight-\fold);
\node[anchor=center, text width=\filewidth cm, align=center]
at (\filewidth/2,\fileheight/2) {DocBook};
\end{tikzpicture}
};

% ---------------- HTML ----------------
\node (html) [right=of xslt] {
\begin{tikzpicture}[x=\scale cm,y=\scale cm]
\draw[fill=white]
(0,0) -- (0,\fileheight) --
(\filewidth-\fold,\fileheight) --
(\filewidth,\fileheight-\fold) --
(\filewidth,0) -- cycle;
\draw[fill=white]
(\filewidth-\fold,\fileheight) --
(\filewidth-\fold,\fileheight-\fold) --
(\filewidth,\fileheight-\fold) -- cycle;
\node[anchor=center, text width=\filewidth cm, align=center]
at (\filewidth/2,\fileheight/2) {HTML};
\end{tikzpicture}
};

% ---------------- PDF ----------------
\node (pdf) [right=of docbook] {
\begin{tikzpicture}[x=\scale cm,y=\scale cm]
\draw[fill=white]
(0,0) -- (0,\fileheight) --
(\filewidth-\fold,\fileheight) --
(\filewidth,\fileheight-\fold) --
(\filewidth,0) -- cycle;
\draw[fill=white]
(\filewidth-\fold,\fileheight) --
(\filewidth-\fold,\fileheight-\fold) --
(\filewidth,\fileheight-\fold) -- cycle;
\node[anchor=center, text width=\filewidth cm, align=center]
at (\filewidth/2,\fileheight/2) {PDF};
\end{tikzpicture}
};

% ---------------- XSL-FO (ECHT ZENTRIERT) ----------------
\node (fo) at ($(xslt)!0.5!(docbook)$) {
\begin{tikzpicture}[x=\scale cm,y=\scale cm]
\draw[fill=white]
(0,0) -- (0,\fileheight) --
(\filewidth-\fold,\fileheight) --
(\filewidth,\fileheight-\fold) --
(\filewidth,0) -- cycle;
\draw[fill=white]
(\filewidth-\fold,\fileheight) --
(\filewidth-\fold,\fileheight-\fold) --
(\filewidth,\fileheight-\fold) -- cycle;
\node[anchor=center, text width=\filewidth cm, align=center]
at (\filewidth/2,\fileheight/2) {XSL--FO};
\end{tikzpicture}
};

% ---------------- Pfeile ----------------
\draw[->] (xml) -- (xslt);
\draw[->] (xml) -- (docbook);
\draw[->] (xslt) -- (html);
\draw[->] (xml) -- (fo);
\draw[->] (fo) -- (pdf);
\draw[->] (docbook) -- (pdf);

\end{tikzpicture}
\caption{Übersicht der Dokumenttypen und Transformationspfade}
\label{fig:files}
\end{figure}






\section{Datensatz}





\section{Transformationen}





\section{Programmierung}





\section{Zusätzliche Informationen}

\begin{tabular}{|l|l|}
    \hline
    Products.xml & XML-Datenquelle mit Produktinformationen \\ \hline
    Products.xsd & XML-Schema zur Validierung der Produktdaten \\ \hline
    Products.xslt & XSLT-Stylesheet zur Transformation von XML nach HTML \\ \hline
    Products.html & Ergebnisdokument im HTML-Format \\ \hline
    Products.xsl & XSL-FO-Stylesheet zur Erzeugung von Layout/Formatierung \\ \hline
    Products.fo & Zwischenformat mit XSL-FO für PDF-Erstellung \\ \hline
    Products.pdf & Enddokument im PDF-Format \\ \hline
    Products\_DocBook.xml & XML-Dokument im DocBook-Format \\ \hline
    Products2.fo & Zweites XSL-FO-Dokument für alternative PDF-Ausgabe \\ \hline
    Products2.pdf & Zweite PDF-Ausgabe basierend auf Products2.fo \\ \hline
\end{tabular}



\end{document}
